\documentclass[twoside, 11pt]{article}

\pagestyle{empty}

\author{Conor McGann}
\title{PlanWorks Requirements Specification - FY03}

\begin{document}

\maketitle

\section{Introduction}
This document outlines the requirements to be addressed in FY2003, the first year, for the PlanWorks project. The main elements of this work are:
\begin{enumerate}
\item A new Domain Description Language to make modeling more productive
\item A Plan Visualization Tool.
\item A Planning Debugger, for analysis of planner traces.
\item A Test Harness to manage regression testing of planners based on output.
\item An Application Framework in which the various tools can be integrated.
\end{enumerate}

\section{New Domain Description Language (NDDL)}
\begin{enumerate}
\item A domain description language must be provided that is at least functionally equivalent to the current DDL. 
\item A translator must be provided to translate DDL to NDDL without changing the semantics of the model.
\item A parser must be provided to parse and validate models written in NDDL.
\item A model loader must be provided to ingest NDDL into the EUROPA Model Manager.
\item The existing EUROPA initial state language must continue to be supported when using onl DDL equivalent functionality of NDDL.
\item An initial state description langauge must be provided that has the full power of NDDL.
\item A loader must be provided to ingest problem definitions written in the new problem definition language.
\item Exisitng initial state descriptions for DDL models must be translated into the new state description language.
\item A users guide must be provided to explain the feautures of NDDL and the Initial State language. 
\end{enumerate}

\section{Application Framework}
\begin{enumerate}
\item The PlanWorks Application Framework shall integrate the following tools:
\begin{enumerate}
\item Plan Viz - for visualization of single Partial Plans.
\item Test Case Manager - for capture and execution of planner regression tests.
\item Plan Debugger - for analysis of planning sequence output from a planner to troubleshoot problems.
\end{enumerate}
\item The system shall allow users to manipulate projects:
\begin{enumerate}
\item Create Project
\item Open Project
\item Delete Project
\end{enumerate}
\item The system shall require the following core data to be stored for each project:
\begin{enumerate}
\item The name of the project. 
\item Any location or connection information for the Repository to be used for the Project.
\end{enumerate}
\item The system shall allow storage of an tool specific data required for this project.
\item The system shall allow visualization of project data.
\item The system shall allow all Tools to be started and controlled from the Project.
\item The system shall prohibit use of any Tools unless a Project is open.
\item The system will be started with no Project open.
\item At most one Project may be open at any time.
\end{enumerate}

\section{Data logging from EUROPA}
\begin{enumerate}
\item Must support logging of the following events from EUROPA to a PlanWorks Repository:
\begin{enumerate}
\item \begin{verbatim}BEGIN_PROPAGATION\end{verbatim}
\item \begin{verbatim}END_PROPAGATION\end{verbatim}
\item \begin{verbatim}TOKEN_INSERTED\end{verbatim}
\item \begin{verbatim}TOKEN_FREED\end{verbatim}
\item \begin{verbatim}TOKEN_CREATED\end{verbatim}
\item \begin{verbatim}CONSTRAINT_INSERTED\end{verbatim}
\item \begin{verbatim}CONSTRAINT_DELETED\end{verbatim}
\item \begin{verbatim}CONSTRAINT_EXECUTED\end{verbatim}
\item \begin{verbatim}VARIABLE_CREATED\end{verbatim}
\item \begin{verbatim}VARIABLE_DELETED\end{verbatim}
\item \begin{verbatim}VARIABLE_DOMAIN_EMPTIED\end{verbatim}
\item \begin{verbatim}VARIABLE_SPECIFIED_DOMAIN_CHANGED\end{verbatim}
\item \begin{verbatim}VARIABLE_DERIVED_DOMAIN_CHANGED\end{verbatim}
\item \begin{verbatim}NON_UNIT_DECISION_MADE\end{verbatim}
\item \begin{verbatim}UNIT_DECISION_MADE\end{verbatim}
\item \begin{verbatim}DECISION_RETRACTED\end{verbatim}
\end{enumerate}

\item Must allow for run-time control of event logging fidelity.
\item Must allow for a PartialPlan to be output at any time during planning at the request of the Planner, as long as the Plan Database is not inconsistent.
\item Must support run-time control of logging output destinations.
\end{enumerate}

\section{Plan Visualization}
\begin{enumerate}
\item A Project shall contain 0 or more named planning sequences.
\item A Planning Sequence shall contain 1 or more Steps.
\item Each Step of a Planning Sequence must contain a Partial Plan.
\item Each step of a Planning Sequence may contain one or more Transactions.
\item All Partial Plans in a Planning Sequence share the same Model.
\item The Partial Plan of the first Step is the initial state of a planning process.
\item The system must permit display of any Partial Plan in a Planning Sequence.
\item The system must restrict display of multiple instances of the same Partial Plan.
\item The system will allow multiple Partial Plans to be visualized at once.
\end{enumerate}
\subsection{Content Specification}
\begin{enumerate}
\item The user must be able to define a Content Specification which will identify a subset of content in a given partial plan.
\item The user must be able to restrict content according to:
\begin{enumerate}
\item Timeline
\item Predicate
\item Time Interval
\item Constraint
\item Variable Type - Temporal, Object, Reject, Parameter
\end{enumerate}
\item A variable is excluded if its token is excluded.
\item A constraint is excluded if all of its variables are excluded.
\item At most one Content Specification will apply for a Partial Plan.
\item The user may edit the current Content Specification of a Partial Plan at any time.
\item The user must explicitly apply any changes in a Content Specification for them to take affect. Once applied, their effects will be immediate.
\item A Content Specifcation may be saved in a Project.
\item A stored Content Specification may be selected and re-applied to a Partial Plan.
\end{enumerate}

\subsection{Views}
\begin{enumerate}
\item A view applies to exactly one Partial Plan.
\item All views in PlanViz are restricted to data described by the active Content Specification. If no Content Specification is provided, all data in the Partial Plan is used.
\item The collection of views to be defined in PlanViz must support the following queries:
\begin{enumerate}
\item What are the Tokens by:
\begin{enumerate}
\item Object
\item Timeline (including no Timeline)
\item Predicate
\item MasterOf
\item SubgoalOf
\item Time Interval
\item Slot
\end{enumerate}
\item What are the Variables by:
\begin{enumerate}
\item Token
\item Constraint
\item Time Interval
\item Bound vs. Unbound
\end{enumerate}
\item What are the Constraints by:
\begin{enumerate}
\item Variable
\item Token
\end{enumerate}
\item No two instances of the same view may be open if they share the same content specification. For now this means we are restricting ourselves to at most one instance of each view at once.
\item Barring other restrictions alread defined, it msut be possible to open multiple views of the same partial plan at once.
\item The user must be able to open, close, re-size and position all views.
\item Changes to the Content Specification of a View take affect as soon as they are applied by the user.
\end{enumerate}
\end{enumerate}
\subsection{Planner Debugger}
\subsection{Test Harness}
\end{document}



